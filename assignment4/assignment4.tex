\documentclass[journal,12pt,onecolumn,]{IEEEtran}

\usepackage{hyperref}
\usepackage{pgf}
\usepackage{booktabs}
\usepackage{wasysym}


%%%%%Prof Zhou's Rant on Citations, Punctuation, and Quotes %%%%%%
%%%
%%% Yes - ...of seventy-two''~\cite{42}.
%%% Yes - ...of seventy-two.''\footnote{42}
%%%
%%% NO! - ...of seventy-two.''~\cite{42}.
%%%
%%% https://advice.writing.utoronto.ca/wp-content/uploads/sites/2/quotations.pdf
%%%
%%%%%%%%%%%%%%%%%%%%%%%%%%%%%%%%%%%%%%%%%%%%%%%%%%%%%%%%%%%%%%%%%%

\begin{document}

\title{Evaluation of 2017 Open Source Survey Results \\
{\normalsize ECE1785: Assignment 4 - Experimental Design}}

\author{Amos Hebb, Yilong Wang, Zhenyue Yu\\ \small University of Toronto}

\maketitle

\section{Introduction}

In this assignment, we analyze the \texttt{github.com} 2017 Open Source Survey~\cite{gitHubOpenSourceSurvey2017},

\section{Hypothesis}

$H_0$: Uniquely feminine names have no impact on the rate of experiencing unwelcoming language.

$H_1$: Github profiles with uniquely feminine names experience unwelcoming language at a greater frequency.

\subsection{Motivation}

After conducting a quantitative analysis of the 2017 open-source survey outcomes, we discovered that female respondents reported a higher likelihood of receiving unwelcoming content compared to male respondents. However, we noted that the gender distribution of the survey sample was highly imbalanced, with only 3.5\% female respondents in our cleaned dataset, which raises concerns about the generalizability of the findings.

To address this gap in the literature, we conducted a follow-up study to test the hypothesis that encountering unwelcoming language on the platform is affected by gender. Our research question is essential in addressing potential gender disparities on the platform, and its significance is amplified by the imbalanced gender distribution in the survey results. We utilized a controlled experiment design to ensure the validity and reliability of the results, and carefully selected a diverse sample of participants to ensure the generalizability of the findings.

By exploring the impact of gender on the likelihood of encountering unwelcoming language on the platform, our study provides valuable insights into potential sources of gender disparities in the open-source software development community. Our findings may help inform strategies to promote diversity and inclusivity in the industry, and shed light on the broader societal issue of gender-based discrimination in technology.


\section{Participants}
In this experiment, a control group and an experimental group will be utilized to explore whether encountering unwelcoming language on the GitHub platform is affected by having a uniquely feminine name. The control group will consist of participants with a GitHub profile featuring a uniquely masculine name, while the experimental group will have a uniquely feminine name. Both groups will be identical in every other aspect, including profile information, profile picture, and pull request content. The aim of the control group is to establish an A-test for occurrence of unwelcoming language in pull request responses on GitHub. In comparison, the experimental group will enable researchers to determine whether encountering unwelcoming language is influenced by the gendered nature of the name associated with the profile. Through comparing the occurrence of unwelcoming language in pull request responses between the control and experimental groups, the experiment aims to determine whether there is a significant difference based on the gendered nature of the name.
\subsection{Control Group}
control (gets PR from 'male' dummy). Justify this by saying that 9n\% of people in the ghsurvey~\cite{gitHubOpenSourceSurvey2017} were men and therefore it is okay to treat male coded gh profiles as control.

\subsection{Experimental Group}
zTODO: Describe experimental (gets PR from 'female' dummy) vs

\subsection{Recruitment}

aTODO: Oh yikes, we can't get prior consent. We randomly sample from response. It's easy to recruit because we are just randomly firing out PRs. This is good because blah blah blah. This is ethical becaule blah blah blah (mention minnesota), We will not record usernames of any responses and remove any identifying language from recorded response, blah blah blah.

\section{Experimental Procedure}

aTODO: fluff this up bigtime

\subsection{Sample Size}

We will create 208 pull requests of equal quality, and 208 github profiles, 104 with uniquely feminine names, 104 with uniquely masculine names.
208 is calculated based on the formula \ref{nsamples} for sample size calculation for comparing proportions~\cite{wang2007sample}.

\begin{equation} \label{nsamples}
    n \ge ( Z_{\alpha/2} + Z_{\beta} )^2 \times ( p_{\male} ( 1 - p_{\male} ) + p_{\female} ( 1 - p_{\female})) / ( p_{\male} - p_{\female})^2
\end{equation}

For a confidence interval of 95\%, $Z_{\alpha/2} = 1.96$, for a power of 80\% $Z_{\beta} = 0.84$.
Based on preliminary results from assignment 3, proportion of men reporting unwelcoming language $p_{\male} = 0.15$,
while women reporting unwelcoming language $p_{\female} = 0.26$.

\begin{equation}
    208 \ge (1.96 + 0.84)^2 \times (0.15 \times 0.85 + 0.26 \times 0.74) / (0.15 - 0.26)^2
\end{equation}

\subsection{Curating Uniquely Female Names}

To decide on which names are uniquely feminine, we use first name frequency data calculated from the US Baby Names.~\cite{names}
\citet{geiger17} show that most github users are between 18 and 44 years old.
We group US Baby names by sex and count occurances between 1979 and 2005.
We select the 1000 most popular names by sorting by the sum of occurances across sex.
We calculate then sort by the ratio between male and female occurances.
We select the 104 head rows to represent the most uniquely female popular names, the 104 tail rows to represent the most uniquely male popular names.

As a sanity check, 10 names were sampled from each list and a small survey is conducted to see if random participants would identify each name as "Feminine", "Masculine", "Other", or "Can't Tell".
The complete lists, as well as the results of this survey will be made available an appendix.

Both the Feminine and Masculine name lists are shuffled using the same random seed.
These are then paired with the 104 most common surnames from 1979 to 2005 such that every surname is given a uniquely feminine and uniquely masculine name of similar popularity.

\subsection{Creating Github Profiles}

The results in the GitHub Open Source Survey may have had confounding indicators of group membership which may have reduced occurances of unwelcoming language.
To ensure we answer our research question, the github profiles must be identical other than treatment group having a uniquely feminine name.
For our experiment, minimal new profiles will be created, and all will be opening their first commit.
Github generated profile photo, no bio, no readme, no github page, no green squares, no badges, no pinned or starred repos.
Usernames and emails will be full name followed by 4 random digits.
Further customization is not needed as profiles like this are not unusual in open source.
One example, at the time of writing the top two non-bot contributors to \url{\https://github.com/rust-lang/rust-clippy} use generated icons and `first-last' usernames with no bio or readme.

\subsection{Curating Pull Requests} %acquiring or preparing any materials we need

Because the intention of this experiment is to sample a broad array of communities across GitHub and create many pull requests of equal quality, pull requests must be language agnostic and fairly trivial to create and review.

We are also aware of the negative impact on the open source community a University of Minnesota experiment TODO:cite involving creating pull requests has caused in the past.
To ensure a positive impact on the open source community, all pull requests will require minimal maintainer time, being fewer than 100 insertions or deletions to a single file.
They will not introduce security vulnerabilities, by not modifying source code at all.
They will be useful improvements to documentation that have been explicitly solicited by the projects in question.
All recommendations from maintainers will be implemented in a timely manner, all issues will be either merged or closed within 3 weeks of opening.

We find active projects by using GitHub search to find the 750 most popular projects with an open issue containing the text "README" in any programming language, English spoken language projects.
From this list, researchers each started looking at projects and started addressing simple issues.
Remove secrets, sync readme with JSDocs or requirements.txt, Bump docker version, and similar type fixes.
Researchers intentionally avoided opinionated issues, any requests asking for recommendations, emphasizing advantages, or prescribing solutions were not addressed.

208 pull requests of roughly equal depth, quality, and utility were authored.
Only after authoring were all pull requests randomly shuffled then assigned to profiles to ensure no bias is introduced while writing pull requests.

\subsection{Opening 208 Pull Requests}

Using the gh cli, 208 pull requests were all opened within 90 minutes of each other starting at 2pm EST on a Tuesday.
One researcher will monitor for any comments or requests on any of the accounts, and will relay any requests, questions, or followup without profile identification to the two `blind' researchers.
One of these two `blind' researchers will prepare revisions or respond to questions, verify with their colligue of the other gender identity that the response is neutral, before handing back to coordinator to send response from correct account.
After 3 weeks, all unmerged issues are closed.

The `merged',`maintainer-closed',`stale-closed' status of each issue, along with any text responses, are recorded.
In response transcript, maintainer usernames are replaced with `maintainer', all contributing profiles replaced with either `feminine name' or `masculine name'.
Actual contents of pull requests, along with uniquely identifiable responses and code snippits, are omitted.
Researchers coded each response as either `contains unwelcoming language' or `no unwelcoming language'.
Status, responses, and coding are all provided in an Appendix.

\section{Analysis}

Our study aims to investigate whether there are gender differences in the frequency of receiving unwelcoming language in response to pull requests on GitHub. To achieve this, we will utilize a two-step coding process. Specifically, two independent researchers will first code the language used in response to each pull request as either 'contains unwelcoming language' or 'no unwelcoming language.'

To ensure the accuracy and reliability of the coding process, the researchers will follow a set of pre-defined guidelines and be blinded to the gender of the requester. After independently coding the responses, the researchers will discuss their coding results and come to a consensus on whether each response contains unwelcoming language or not. Any discrepancies will be resolved through a third-party review.

If a pull request does not receive any responses, it will be coded as 'no unwelcoming language.' Based on the coding results, we will calculate the ratio of unwelcoming responses received when using an account with a 'feminine' name versus a 'masculine' name. We will calculate the difference between the two ratios and test whether this difference is statistically significant using permutation tests.

In addition, we will also investigate whether the acceptance status of the pull request has any impact on receiving unwelcoming language in response. Specifically, we will calculate the ratio of receiving unwelcoming responses with their corresponding acceptance status, and determine whether this impact is gender-related.


\section{Threats to Validity}

c(a wants to rant some too)TODO: Here we rant about anon responses (who is deltron3000), ambiguous names (who is riley?), these are very trivial PRs which are not as involved, these are all new accounts with no reputation, we may be measuring a `new contributor only' effect.  Whatever, rif a bit, only worth 1 point, don't lose too much sleep here.
Internal validity -> coding of unwelcoming, selection? sample size may be too small
external validity -> generalize: encountering unwelcoming language in other part of github or coding related communities
\section{Conclusion}
%

\bibliographystyle{IEEEtran}
\bibliography{assignment4}

\end{document}
\end{document}
\end{document}
\end{document}
