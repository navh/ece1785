\documentclass[journal,12pt,onecolumn,]{IEEEtran}
\usepackage{hyperref}

%%%%%Prof Zhou's Rant on Citations, Punctuation, and Quotes %%%%%%
%%%
%%% Yes - ...of seventy-two''~\cite{42}.
%%% Yes - ...of seventy-two.''\footnote{42}
%%%
%%% NO! - ...of seventy-two.''~\cite{42}.
%%%
%%% https://advice.writing.utoronto.ca/wp-content/uploads/sites/2/quotations.pdf
%%%
%%%%%%%%%%%%%%%%%%%%%%%%%%%%%%%%%%%%%%%%%%%%%%%%%%%%%%%%%%%%%%%%%%

\begin{document}

\title{Use of Theory\\
{\normalsize ECE1785: Assignment 2 - Theory}}

\author{Amos Hebb, Yilong Wang, Zhenyue Yu\\ \small University of Toronto}

\maketitle

%Did the theory play a major role in framing the research questions?

%Was the methodology influenced by the theory?

%How did the theory impact the data that was collected and the way it was analyzed?

%How important was the theory to this research? Does it help explain the results? Does it help generate new questions?


\section{Introduction}

Both papers leverage Rogers' diffusion of innovation (DOI) theory. This theory outlines ``the process by which an innovation is communicated through certain channels over time among the members of a social system''~\cite{rogers1995attributes}.

\section{Heard it through the {\sc Git}vine}

Theory played a central role in shaping the research questions in ``Heard it through the {\sc Git}vine: an empirical study of tool diffusion across the npm ecosystem~\cite{lamba2020heard}.''. In Section 2, the research question being explored is ``How do the four other important attributes of innovations impact their diffusion?''

These attributes are observability, relative advantage, compatibility with the adopter's context, and relative complexity, all of which are named among the five attributes of innovations that influence adoption, according to Roger's theory. The only missing attribute is trialability, which is considered to be indistinguishable among different tools.

These examples show how the theory was instrumental in forming the research questions. |Subject matter in the question is derived from the theory. In other words, the research question in this paper was structured around the theory.

\subsection{Influence on Methodology due to Theory}

A central component of the methodology is hypothesis testing. It was observed that Roger's theory played a crucial role in the formation of the hypotheses.

For instance, Hypothesis H\textsubscript{1}  posits that `explore' adopters drive diffusion, which is a category in Roger's theory for classifying adopters. Hypothesis
H\textsubscript{5} examines tool compatibility and H\textsubscript{7} examines the impact of complexity on the diffusion of tools. Both the objects of study for these hypotheses belong to the attributes of innovation in Roger's theory.
In Section 4, the authors also mention that badges were selected as proxies for tool use because they have higher observability, which is another attribute mentioned in the theory.

The theory had a strong impact on methodology, as it informed the development of hypotheses as well as the choice of the study objects.

%one of:
\subsection{Results support Theory}

The results of this paper show that Rogers' DOI theory holds for software engineering tools. Differences between early and late adopters exist among developers in a way that conforms to DOI. Regressions demonstrate that many aspect hold, there is particularly strong evidence for some aspects like social networks and visibility and less evidence for complexity.

%%%%%%%%%%%% PAPER 2

\section{Social Influences on Secure Development Tool Adoption}

``Social influences on secure development tool adoption: why security tools spread''~\cite{xiao2014social}. focuses on the social aspects of DOI, especially how DOI focuses on communication.
DOI also includes technological and organizational factors.
There are other theories intentionally not considered: Technology Acceptance Model, Perceived Characteristics of Innovation, Theory of Planned Behaviour, and Model of Personal Computer Utilization.

DOI theory plays a major role in framing the research question.
In the present paper, the authors analyze the role of social influences on the usage of secure development tools and investigate why these tools are rarely used today.
The authors make use of the DOI theory to approach the research question comprehensive and principled manner.
With a focus on DOI theory, the literature review of the paper discusses existing theories on technology diffusion and prior studies on developers adopting tools and technologies.
Therefore, DOI theory contributes to the understanding of the role social influence plays in the adoption of secure development tools.


\subsection{Influence on Methodology due to Theory}

The methodology of the paper was also influenced by DOI theory.
In the study, qualitative interviews were conducted to understand factors that affect adoption or rejection decisions, and how these factors interact.
To gauge the importance of social and technological factors from DOI theory, more direct questions were included in the interview to elicit information about participants' security tool adoption choices.
The DOI theory outlines how social and technological factors affect adoption decisions, which the authors illustrate through the findings from the interviews.

\subsection{Theory Supports Slow Diffusion}

Users, even after being exposed to tools or even forced to use these tools, reported only using them internally or stopping using them out of ``laziness'' or reasons they could not recall at some point. While in one case peer recommendation happened, this was not typical. While this does not directly demonstrate DOI theory, the theory was helpful in this context to systematically look for factors contributing to the remarkably slow diffusion of security tools.

\section{Conclusion}

Borrowing well-studied theories from the social sciences provide a solid grounding for conducting empirical research in computer engineering. In some contexts, like \textit{npm} tool use, these theories hold and can be used to provide actional recommendations about how to promote tools. In other cases where theories do not as plainly explain findings, the structured approach helps inform methodology.

\bibliographystyle{IEEEtran}
\bibliography{assignment2}

\end{document}
