\documentclass[journal,12pt,onecolumn,]{IEEEtran}
\usepackage{hyperref}

%%%%%Prof Zhou's Rant on Citations, Punctuation, and Quotes %%%%%%
%%%
%%% Yes - ...of seventy-two"~\cite{42}.
%%% Yes - ...of seventy-two."\footnote{42}
%%%
%%% NO! - ...of seventy-two."~\cite{42}.
%%%
%%% https://advice.writing.utoronto.ca/wp-content/uploads/sites/2/quotations.pdf
%%%
%%%%%%%%%%%%%%%%%%%%%%%%%%%%%%%%%%%%%%%%%%%%%%%%%%%%%%%%%%%%%%%%%%

\begin{document}

\title{Use of Theory\\
{\normalsize ECE1785: Assignment 2 - Theory}}

\author{Amos Hebb, Yilong Wang, Zhenyue Yu\\ \small University of Toronto}

\maketitle

%Did the theory play a major role in framing the research questions?

%Was the methodology influenced by the theory?

%How did the theory impact the data that was collected and the way it was analyzed?

%How important was the theory to this research? Does it help explain the results? Does it help generate new questions?


\section{Introduction}

\section{Heard it through the {\sc Git}vine}

Heard it through the {\sc Git}vine: an empirical study of tool diffusion across the npm ecosystem~\cite{lamba2020heard}.

\subsection{Theory}
The theory played a central role in shaping the research questions for this paper. As noted in Section 2, the research question being explored is``How do the four other important attributes of innovations impact their diffusion?"
 These attributes are observability, relative advantage, compatibility with the adopter's context, and relative complexity, all of which are named among the five attributes of innovations that influence adoption, according to Roger's theory. The only missing attribute is trialability, which is considered to be indistinguishable among different tools. 
 Therefore, it is believed that the theory was instrumental in forming the research questions, as the subject matter in the question is derived from the theory. In other words, the research question in this paper was structured around the theory.

\subsection{Influence on Methodology due to Theory}
A central component of the methodology is hypothesis testing. It was observed that Roger's theory played a crucial role in the formation of the hypotheses.
For instance, Hypothesis H\textsubscript{1}  posits that explore adopters drive diffusion, which is a category in Roger's theory for classifying adopters.
H\textsubscript{5} examines tool compatibility and H\textsubscript{7} exams the impact of complexity on tool diffusion. Both the objects of study for these hypotheses belong to the attributes of innovation in Roger's theory. 
In Section 4, the authors also mention that badges were selected as proxies for tool use because they have higher observability, which is another attribute mentioned in the theory.
Thus, it can be concluded that the theory has a strong impact on the methodology, as it informs the development of hypotheses and the choice of the study objects.

\subsection{Theory's impact on data collection and analysis}
To some extent, the theory also influence the data collection and analysis in the first paper. The data collected here includes the badges taken by various npm packages, and the badges were selected because of their perceived high observability, which is a crucial element of innovation in Rogers' theory.
While constructing the networks for analysis, the author identified early adopters using standard deviation, a method suggested by Roger's theory.
When analyzing the data, the authors applied a logistic regression model to validate Hypothesis H\textsubscript{1} and used survival models for the other hypotheses. The differening choice of methods was to accommodate the fact that the hypotheses were studying different elements or attributes in Roger's theory.
\subsection{Theory Explains Results}
The theory is also critical in providing an explanation for the results. According to the analysis, the authors found that six of the seven hypotheses they proposed
stand, which, in return, support the theory's explanation on some aspects of the diffusion of tools in npm system. While discussing the exceptions in Hypothesis H\textsubscript{6},
the authors pointed out two factors, more features, which could be interpreted as more advantages, and more distinct badges that suggesting higher observability.The authors appear to have presented their findings from the perspective of the theory.
What's more, the theory also helps researchers to identify new questions, which focus on two elements of innovation that they do not explore thoroughly in this paper, trialability and observability.
In conclusion, we believe that the theory was instrumental in providing an explanation for the results and generating new questions for future research.
\subsection{Results do not Support Theory} %At least one paper is this one.

%%%%%%%%%%%% PAPER 2

\section{Social Influences on Secure Development Tool Adoption}

Social influences on secure development tool adoption: why security tools spread~\cite{xiao2014social}.


\subsection{Theory}

Rogers' diffusion of innovation (DOI) theory.
"the process by which an innovation is communicated through certain channels over time among the members of a social system"~\cite{rogers1995attributes}.
This paper focuses only on social aspects, especially how DOI focuses on communication.
DOI also includes technological and organizational factors.
There are other theories intentionally not considered: Technology Acceptance Model, Perceived Characteristics of Innovation, Theory of Planned Behaviour, and Model of Personal Computer Utilization.


\subsection{Influence on Methodology due to Theory}

\subsection{Theory's impact on data collection and analysis}

%one of:
\subsection{Theory Explains Results}

\subsection{Results do not Support Theory} %At least one paper is this one.

\bibliographystyle{IEEEtran}
\bibliography{assignment2}

\end{document}
