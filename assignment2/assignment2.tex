\documentclass[journal,12pt,onecolumn,]{IEEEtran}
\usepackage{hyperref}

%%%%%Prof Zhou's Rant on Citations, Punctuation, and Quotes %%%%%%
%%%
%%% Yes - ...of seventy-two"~\cite{42}.
%%% Yes - ...of seventy-two."\footnote{42}
%%%
%%% NO! - ...of seventy-two."~\cite{42}.
%%%
%%% https://advice.writing.utoronto.ca/wp-content/uploads/sites/2/quotations.pdf
%%%
%%%%%%%%%%%%%%%%%%%%%%%%%%%%%%%%%%%%%%%%%%%%%%%%%%%%%%%%%%%%%%%%%%

\begin{document}

\title{Critique on the Use of Theory\\
{\normalsize ECE1785: Assignment 2 - Theory}}

\author{Amos Hebb, Yilong Wang, Zhenyue Yu\\ \small University of Toronto}

\maketitle

%Did the theory play a major role in framing the research questions?

%Was the methodology influenced by the theory?

%How did the theory impact the data that was collected and the way it was analyzed?

%How important was the theory to this research? Does it help explain the results? Does it help generate new questions?


\section{Introduction}
In this assignment, we studied two papers. One is Lamba et al~\cite{lamba2020heard}, and the other is Xiao et al~\cite{xiao2014social}. The main theory behinds the two papers here is Roger's Diffusion of Innovation (DOI) theory~\cite{rogers1995attributes}. 
Although they share the same theory, the two papers applied different research methods to analyze their questions. They also have different adoptions of the theory. In the first paper, the \textit{social system} is GITHUB, while the second paper treats a group of developers as social system.
The first paper chose to use quantitative methods, while the second paper conducted a qualitative analysis. Nonetheless, the theory played a major role in both of the papers in many aspects.
%\section{Heard it through the {\sc Git}vine}

%Heard it through the {\sc Git}vine: an empirical study of tool diffusion across the npm ecosystem~\cite{lamba2020heard}.

\section{Theory in the formation of Research Questions}
The theory played a central role in shaping the research questions for the first paper. As noted in Section 2, the paper explored how do the four \textit{other important attributes} of innovations impact their diffusion.
 These attributes are observability, relative advantage, compatibility with the adopter's context, and relative complexity, all of which are named among the five attributes of innovations that influence adoption, according to Roger's theory. The only missing attribute is trialability, which is considered to be indistinguishable among different tools. 
 Therefore, it is believed that the theory was instrumental in forming the research questions, as the subject matter in the question is derived from the theory. In other words, the research question in this paper was structured around the theory.

The DOI theory also plays a major role in framing the research question in the second paper. This paper aim to analyze the role of social influences on the usage of secure development tools and to investigate why these tools are spread. DOI theory explains how social factors such as organizational factors and communication channel factors in which innovations diffuse affect their adoption, 
which forms the main part of research question. Therefore, DOI theory contributes to the understanding of the role social influence plays in the adoption of secure development tools. 
\section{Influence on Methodology due to Theory}
For the first paper, a central component of the methodology is hypothesis testing. It was observed that Roger's theory played a crucial role in the formation of the hypotheses.
For instance, Hypothesis H\textsubscript{1}  posits that explore adopters drive diffusion, which is a category in Roger's theory for classifying adopters.
H\textsubscript{5} examines tool compatibility and H\textsubscript{7} exams the impact of complexity on tool diffusion. Both the objects of study for these hypotheses belong to the attributes of innovation in Roger's theory. 
In Section 4, the authors also mention that badges were selected as proxies for tool use because they have higher observability, which is another attribute mentioned in the theory.
Thus, it can be concluded that the theory has a strong impact on the methodology, as it informs the development of hypotheses and the choice of the study objects.

Similarly, the methodology of the second paper was influenced by the DOI theory. As mentioned in section Interview Methodology, this paper conducted an interview which included questions about the importance of social and technological factors from DOI theory. 
A qualitative analysis program was used to evaluate the interview records. The DOI theory outlines how social factors affect adoption decisions, which the authors attempted to illustrate through the findings from the interviews. Therefore, DOI helps interview method form the interview questions and helps evaluate the interview records.
\section{Theory's impact on data collection and analysis}
To some extent, the theory also influence the data collection and analysis in the first paper. The data collected here includes the badges taken by various \textit{npm} packages, and the badges were selected because of their perceived high observability, which is a crucial attribute of innovation in Rogers' theory.
While constructing the networks for analysis, the author identified early adopters using standard deviation, a method suggested by Roger's theory.
When quantitatively analyzing the data, the authors applied a logistic regression model to validate Hypothesis H\textsubscript{1} and used survival models for the other hypotheses. The differening choice of methods was to accommodate the fact that the hypotheses were studying different elements or attributes in Roger's theory.

The theory has more direct impact on the second paper. As part of the data collection process, the theory shaped the interview script and the specific questions that were asked. Furthermore, this interview includes directed questions about social and technological factors from DOI theory in addition to open-ended questions. 
DOI theory provided a basis for providing relevant information during the interview. The interview results were analyzed by opening code based on a qualitative data analysis program. In the analysis, the initial code set was derived from Rogers' discussion of DOI, indicating the theory served as a starting point. 
Participant adoption criteria, which were based on regular use for at least one week, were also influenced by DOI theory.
\section{Theory's influence on Results and Future Questions}
For the first paper, the theory is also critical in providing an explanation for the results. According to the analysis, the authors found that six of the seven hypotheses they proposed
stand, which, in return, support the theory's explanation on some aspects of the diffusion of tools in \textit{npm} system. While discussing the exceptions in Hypothesis H\textsubscript{6},
the authors pointed out two factors, more features, which could be interpreted as more advantages, and more distinct badges that suggesting higher observability. The authors appear to have presented their findings from the perspective of the theory.
What's more, the theory also helps researchers to identify new questions, which focus on two attributes of innovation that they do not explore thoroughly in this paper, trialability and observability.
Thus, we believe that the theory was instrumental in providing an explanation for the results and generating new questions for future research.

DOI theory also plays an important role in the second paper. DOI describes the spread of ideas and technologies through a population. This explains the research purpose which aims to investigate social factors that affect developers’ adoption decisions. 
The interview results suggest that developers' social environments and the channels through which tools are communicated affect their adoption of security tools. Since DOI accounts for three types of communication channels, it explains why the communication channel factors could influence developers’ adoption behaviors. 
This paper suggests that an observational study might be better suited for evaluating the temporal aspects of DOI theory and its relationship to security tool adoption since retroactive interviews could not adequately capture these aspects. Furthermore, this paper mentions that a qualitative interview study of developers who write open-source software will help determine if their findings generalize to developers outside of industry. 
They also suggest that a survey study would be helpful in evaluating and quantifying their findings and result in validated recommendations for managers seeking to increase security tool adoption in their organizations.
\section{Conclusion} %At least one paper is this one.
In conclusion, despite the different adoptions of the theory and choices of research methods, the theory was critical for the development of the two papers.
As Creswell~\cite{creswell2017research} said, the theory is like a ``perspective lens" that navigates us through the paper. It provides a solid grounding for conducting empirical research, supplies recommendations for forming research questions and methods,
and helps explain the results while generating new questions. 
%%%%%%%%%%%% PAPER 2

% \section{Social Influences on Secure Development Tool Adoption}

% Social influences on secure development tool adoption: why security tools spread~\cite{xiao2014social}.


% \subsection{Theory}

% Rogers' diffusion of innovation (DOI) theory.
% "the process by which an innovation is communicated through certain channels over time among the members of a social system"~\cite{rogers1995attributes}.
% This paper focuses only on social aspects, especially how DOI focuses on communication.
% DOI also includes technological and organizational factors.
% There are other theories intentionally not considered: Technology Acceptance Model, Perceived Characteristics of Innovation, Theory of Planned Behaviour, and Model of Personal Computer Utilization.


% \subsection{Influence on Methodology due to Theory}

% \subsection{Theory's impact on data collection and analysis}

% %one of:
% \subsection{Theory Explains Results}

% \subsection{Results do not Support Theory} %At least one paper is this one.

\bibliographystyle{IEEEtran}
\bibliography{assignment2}

\end{document}
