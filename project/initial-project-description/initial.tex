\documentclass[journal,12pt,onecolumn,]{IEEEtran}
\usepackage{hyperref}

%%%%%Prof Zhou's Rant on Citations, Punctuation, and Quotes %%%%%%
%%%
%%% Yes - ...of seventy-two''~\cite{42}.
%%% Yes - ...of seventy-two.''\footnote{42}
%%%
%%% NO! - ...of seventy-two.''~\cite{42}.
%%%
%%% https://advice.writing.utoronto.ca/wp-content/uploads/sites/2/quotations.pdf
%%%
%%%%%%%%%%%%%%%%%%%%%%%%%%%%%%%%%%%%%%%%%%%%%%%%%%%%%%%%%%%%%%%%%%

\begin{document}

\title{Distributed CAD\\
{\normalsize ECE1785: Initial Project Description}}

\author{Amos Hebb, Yilong Wang, Zhenyue Yu\\ \small University of Toronto}

\maketitle

%Did the theory play a major role in framing the research questions?

%Was the methodology influenced by the theory?

%How did the theory impact the data that was collected and the way it was analyzed?

%How important was the theory to this research? Does it help explain the results? Does it help generate new questions?

\section{Introduction}

% In the Age of Collaboration, the Computer-Aided Design Ecosystem is Behind: An Interview Study of Distributed CAD Practice 
% Kathy CHENG et al.

% TODO: REMOVE REMOVE REMOVE REMOVE REMOVE REMOVE REMOVE REMOVE REMOVE REMOVE
\section{TODO: REMOVE Summary of Cheng et al.}

Collaboration in CAD is known to be broken.
Aim to close academic-practitioner knowledge gap.
14 challenges identified.
INTRO
Focus on Mechanical CAD.
3 generations: [standalone, distributed, collaborative]
- standalone: single user, single machine
- distributed cad: standalone combined with data managemeent system, shared files.
- synchronous multi-user cad (MUCAD) (different-place, same time)
in 2017: <10\% of pros use cloud-based CAD, >50\% of enterprises not considering
in 2021: 20\% of pros, >50\% hobbyists
interview 20 pros.
RQ1: What are the collaboration challenges faced by professionals using distributed CAD?
RQ2: What strategies are CAD professionals or organizations using to remedy collaboration challenges with distributed CAD?
BACKGROUND
Distributed vs Collaborative, Distributed needs data management tool
data management tools can be [manual, partilaly-automated, fully-automated]
existing probs in literature
- inconsistency in design practices between designers
- `some best practices, standardize cite[53,68]'
- non-interoperability, 
- \textit{dumb solid}, STEP - Standard for the Exchange of Product Data.
- distributed teams suck more
- cross-discipline teams suck more
- IP issues
- each discipline's respective standards, best practices, jargon, and schemata
- git also sucks, here are some attempts [27,34,45,88,89,100,110]
- merge conflicts in git are handled smoothly? (I guess compared to CAD?)
METHODS
semi-structured interviews
20 Pros, friends, CAD primarily. Experience cad: mean 8, sigma 7.
5 aerospace, 4 auto, 4 electronics. (7 unaccounted for?)
CAD Software: 7 use NX, 7 use SolidWorks, 5 use Creo, 10 use >2 regularly (sum!=20 again, not mutually exclusive? Were above? So few participants, why not just be exhaustive?)
Version Control: 4 use manual version control, 16 use fully-automated PLM tools. 3/4 manual in <100 person companies.
See Figure 1 Appendix A. (horizontal JPEG of a table in CM font... I could almost tolerate the single column, but this is just straight up unreadable. This is also far more useful than the written out version.)
Learning: 17 responded? 5 update >monthly, 9 >annual, 3 <annual. (another sum!=20, >20 on 'where')
DATA ANALYSIS
NVivo, open coding.
RESULTS
4 categories, 14 challenges, sorted by pervasive
Cat 4.1) Collaborative Design Challenges and Strategies
C-1) Absent or varied modelling conventions across collaborators
 - manually changing coordinate systems
 - design tolerancing standards different in different teams
 - mechanical/electrical engineering follow different conventions
 - designers fail to follow standards
S-1) Imposing standards and best practices for modelling
Author notes nobody other than ID10 has ever seen this work. 
(ID10 is a CEO, I wonder if his engineers say his extra onboarding docs are as helpful as his VPs tell him they are.)
(amos-thinks: snooze of a suggestion, people said this in code forever, gofmt/black/prettier or die.)
C-2) Infrequent Model Uploads
Checks out, doesn't upload, particularly difficult with big assemblies
No strategy.
(amos-thinks: notice all the `fully-automated PLM' responders say long lived private checkouts exist... interesting)
C-3) Lack of designer awareness of model dependencies
Dep management sucks. Author seems to imply modelling from scratch > dep copy pasta. 
No strategy.
C-4) Lack of awareness of collaborators' model changes
Bad pull discipline. Merge conflicts.
C-5) No Synchronous editing 
0/20 use `3rd gen' cad tools with this ability.
S-2) (Yes, Strategy numbers != challenge numbers in this paper) Using assembly configurations, another worker seems to describe branches (personal sandboxes)
(amos-thinks: sync is really overrated, nobody codes in google docs, I like this suggestion.)
Cat 4.2) Synchronous communication challenges
C-6) Cumbersome presentation of CAD models in synchronous work settings
Design review meetings are awful. Clients make them worse.
C-7) no back-of-the-envelope CAD
Cat 4.3) Data Management Challenges and Strategies
C-8) Lack of interoperability
Different software, even different versions, incompatible. Mech/EE problems x10.
S-3) Use dumb solids. Sounds like CAD practitioners have zero interest. It's a hack.
(amos-thinks: maybe better dumb solids? But frankly I can't see this working, there are no examples anywhere)
S-4) Simplify CAD models. EE people toss rectangles into mech builds.
(amos-thinks: obvious hack. just do twice as much work and keep these in sync magically. Not obvious how this strategy helps.)
C-9) Lack of change summarization support between file versions
Excel spradsheets, tedious. 
C-10) Lack of version control for non-PLM users
check-in/check-out function more like locks on files(!?)
S-5) Just use PLM.
C-11) Poor CAD file organization
S-6) Use standard part libraries
(amos-thinks: This whole para sounds like dependency management strikes again.)
C-12) Poor traceability of scatteed file management 
\textit{redlining}, passing files around with comments. ID20 says aesthetics can't be put it CAD. (dubious, just sounds like trash CAD)
S-7) Using integrated digital markup tools 
(amos-thinks: Yes, this feels solved, just tools suck.)
Cat 4.4) Permissioning Challenges 
C-13) Cumbersome stakeholder access to CAD files. 
External collaborators locked out, fragmentation. 
(amos-thinks: Feels solvable with cloud? Maybe? (some sort of browser-viewer?))
C-14) Lack of stakeholder access to CAD software.
Basically the same thing, but with added SolidWorks vs NXCAD spice. 









\bibliographystyle{IEEEtran}
\bibliography{initial}

\end{document}
