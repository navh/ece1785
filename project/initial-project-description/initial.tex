\documentclass[12pt]{IEEEtran}
\usepackage{cite}
\usepackage{amsmath,amssymb,amsfonts}
\usepackage{algorithmic}
\usepackage{graphicx}
\usepackage{textcomp}
\usepackage{xcolor}
\begin{document}

\title{TODO:Catchy Name\\
{\normalsize ECE1785: Initial Project Description}}

\author{Amos Hebb, Yilong Wang, Zhenyue Yu\\ \small University of Toronto}

\maketitle

\section{Introduction}

\section{Project Description}

A few sentences describing your research question, why it is important, and how it fits into your research (assuming it does).

A few sentences describing the methods you plan to employ, why they are appropriate, and why they make sense together.

The kinds of results you might get and what the contribution would be.

None of these are set in concrete -- they may well change as we learn about more methods, you get feedback from me and fellow students, and you have more time to consider the possibilities. See the complete project description here.

Peer review:
After the group report submission deadline, you will be assigned to three (3) documents created by other teams, and you need to provide feedback.

Provide reasonable and constructive comments. Effectively critiquing the work of others and receiving such criticism is an important communication skill in any technical profession. You need to be honest, direct, and always constructive.

When reading the project description, try to figure out:

- Did the team miss any questions or overlook any ambiguities? 

- Are the RQs reasonable and well-justified?

- Is there any hidden assumptions?

- Do you have any suggestions?

- Do you know any resources that might be helpful to the project?

 

No need to cover all the points above. Remember, you are trying to provide assistance to another team. The objective is not to make yourself look smart, or to look smarter than your colleagues on the other team. The goal is to do your utmost to help them do the best job they possibly can. This includes pointing out the strengths of their work, being very clear about its weaknesses and shortcomings, and providing the most helpful, practical suggestions that you can.

 

Please leave comments on the group report on Quercus.

Your project critique will be evaluated based on:

A clear, honest, direct, and accurate expression of strengths and weaknesses.
Maintaining a constructive tone throughout.

\bibliographystyle{IEEEtran}
\bibliography{initial}

\end{document}
