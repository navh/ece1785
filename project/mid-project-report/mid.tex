\documentclass[sigconf,authorversion,nonacm]{acmart}
\bibliographystyle{ACM-Reference-Format}
\usepackage{hyperref}

\widowpenalties 4 0 9000 10000 10000

%%%%%Prof Zhou's Rant on Citations, Punctuation, and Quotes %%%%%%
%%%
%%% Yes - ...of seventy-two''~\cite{42}.
%%% Yes - ...of seventy-two.''\footnote{42}
%%%
%%% NO! - ...of seventy-two.''~\cite{42}.
%%%
%%% https://advice.writing.utoronto.ca/wp-content/uploads/sites/2/quotations.pdf
%%%
%%%%%%%%%%%%%%%%%%%%%%%%%%%%%%%%%%%%%%%%%%%%%%%%%%%%%%%%%%%%%%%%%%


\begin{document}

\title{Group 25 Interim Report\\Visualizing CAD changes with Boolean Differences}

\author{Amos Hebb}
\affiliation{\institution{University of Toronto}}
\email{a.hebb@mail.utoronto.ca}
\author{Yilong Wang}
\affiliation{\institution{University of Toronto}}
\email{yilongece.wang@mail.utoronto.ca}
\author{Zhenyue Yu}
\affiliation{\institution{University of Toronto}}
\email{zhenyue.yu@mail.utoronto.ca}
% zhenyue Yu <zhenyue.yu@mail.utoronto.ca>; Yilong Wang <yilongece.wang@mail.utoronto.ca>
\makeatletter
\def\@ACM@checkaffil{% Only warnings <<<<<<<<<<<<<<<<
	\if@ACM@instpresent\else
		\ClassWarningNoLine{\@classname}{No institution present for an affiliation}%
	\fi
	\if@ACM@citypresent\else
		\ClassWarningNoLine{\@classname}{No city present for an affiliation}%
	\fi
	\if@ACM@countrypresent\else
		\ClassWarningNoLine{\@classname}{No country present for an affiliation}%
	\fi
}
\makeatother

\maketitle

\section{Introduction}
This section will include the background of our project and the introduction of computer-aided design (CAD), Boolean differences, OpenSCAD and GIT.
\subsection{CAD}
Definition and significance of CAD in complex product design and development. Description of challenges associated with distributed CAD practice.
\subsection{Boolean Differences}
Explanation of boolean differences visualization techniques and how visualizations of boolean differences can be an alternative to current methods for summarizing changes between CAD file versions.
\subsection{OpenSCAD}
Description of OpenSCAD.
\subsection{GIT}
Description of git, git differences, and git cherry-pick.
% \begin{enumerate}
% 	\item Define CAD
% 	\item Describe Distributed CAD according to \citet{cheng2023age}
% 	\item Define summarization
% 	\item Describe \texttt{GIT}, \texttt{git diff}, and \texttt{git cherry-pick}
% 	\item Describe boolean operations on 3D solids.
% 	\item Describe OpenSCAD
% \end{enumerate}

\section{Research Question}

In this section, we will outline our research question and discuss the motivation behind its development.
\section{Literature Review}

The motivating paper for this experiment is \citet{cheng2023age}, the authors have helped point out existing tools like \texttt{onshape.com} which implement similar interfaces.

\subsection{Previous Work}
Summarize the findings from \citet{cheng2023age} related to distributed teams, version control, and summarization.
We will summarize a chapter from \citet{Frazelle_2021} on version control to justify \texttt{GIT} like semantics.

\subsection{Existing CAD Version Control}
Summarize what we will be borrowing from Github's \citet{github_blog_2013}, and the implementation in the change detection solution used by \citet{3drepo_blog}.
We will outline GrabCad's cosmo quiz\cite{revisions_2014}, an attempt to codify current best practices when working with CAD.
We will summarize both Change Management and version control mechanisms in \citet{Bricogne_Rivest_Troussier_Eynard_2012}, and summarize opportunities for improvements identified.

\section{Methodology}

Subsections outlining details of both of our proposed experiments, including refinements made as a result of feedback described in Appendix A.

\subsection{Timed Cherry-Picking Experiment}

We will elaborate on the experiment detailed in the initial proposal, placing emphasis on the evaluation of metrics. In addition to time, we will also incorporate accuracy as a response to the feedback received.

\subsection{Summarization Experiment}

We will provide further details on the experiment, highlighting the specific open-ended questions that will be employed during the interviews in conjunction with this study.

\section{Dataset}

The subsections will cover our carefully curated collection of OpenSCAD files with revisions, as well as their organization. Additionally, we will provide pre-compiled differences visualizations saved in \texttt{.stl} format for review by future researchers. These files will be sourced from the Git commit history of an open-source OpenSCAD project hosted on \texttt{github.com}, and from an open-source project on \texttt{onshape.com}with a complete modification history.
We will also include figures that demonstrate a before, after, addition, and subtraction example for a single, straightforward object.

\section{Results}

% After creating some boolean difference visualizations. These visualizations are the result of taking a \emph{before} and \emph{after} of a single part and creating new solids using boolean operations.
% We will provide our observations of the difference visualizations, and compare to \texttt{onshape.com}'s presentation of modifications.
In this section, we will present and discuss the results we get from each experiment.
\subsection{Timed Cherry-Picking Experiment}
We anticipate observing a difference in both the time taken to complete the task and the accuracy of responses between the experimental group and the control group. If Boolean difference visualizations aid in understanding the changes, we expect the experimental group to require less time and achieve greater accuracy compared to the control group.

\subsection{Summarization Experiment}
We expect to receive a diverse array of responses from the interviewees. The Boolean difference visualizations may influence the choice of words for both the experimental and control groups, even when presented with the same object and asked identical questions. A qualitative analysis of the answers will be conducted to identify any differences.

\section{Discussion}

We will explore our insights concerning the effectiveness of Boolean differences in summarizing changes between CAD files, based on our experiment results, and also discuss any intriguing visual artifacts generated while developing difference visualizations for changes to \texttt{.stl} files.

\section{Future Work}
%Based on preliminary results, either changes to visualizations will be recommended or we will recommend applying for ethics review board approval to evaluate diff tools according to the experiments described.
We will address potential modifications our project may require based on the preliminary findings. These adjustments might encompass alterations to the presentation of Boolean difference visualizations or enhancements to our experimental approaches.

\bibliography{mid}

\appendix
\section{Appendix A: Changes to Project}
%format, bullet points:
% critique
% solution:
\begin{enumerate}
	% \item
	%       \begin{description}
	% 	      \item \textbf{critique - definitions:} Refine the research questions, definition of `summarize' and `understanding' is vague.
	% 	      \item \textbf{solution:} In final report strictly define `summarize' and `understanding' in the context of our research. %#TODO: change research question to something very specific.
	%       \end{description}
	\item
	      \begin{description}
		      \item \textbf{critique - cherry-picking experiment:} Time alone may not accurately reflect the participants' comprehension of the changes made, and the experiment did not take into account the accuracy of the outcomes.
		      \item \textbf{solution:} We have improved the cherry-pick experiment to measure not only the time taken to complete the task, but also the accuracy of the results. For each OpenSCAD object, we will provide participants with a series of \texttt{.stl} files, including the final version and various past versions with committed changes. Participants will receive a plain-language description of some specific local reversions to be made to the object. They will be asked to perform these reversions by selecting which commits to include or exclude, without making arbitrary changes to the object. We will ensure that only one selection of past commits yields the correct result, allowing us to measure the accuracy of their choices.
	      \end{description}
	\item
	      \begin{description}
		      \item \textbf{critique - participant experience:} The proposal did not take the participants' past experience with CAD into consideration.
		      \item \textbf{solution:} We have enhanced our experiments by incorporating a survey prior to the interviews or tasks. Participants will be requested to self-assess their OpenSCAD experience and general CAD skills. Based on their responses, we will categorize the participants accordingly.
	      \end{description}
	\item
	      \begin{description}
		      \item \textbf{critique - object selection:} The proposal did not clarify what type of CAD files the team planned to choose.
		      \item \textbf{solution:} We have refined our CAD file selection to focus on OpenSCAD projects saved in the \texttt{.stl} format. We have discovered a number of public OpenSCAD repositories that house a diverse range of 3D solid objects. Our plan is to curate a carefully chosen collection of \texttt{.stl} files, each with multiple revisions, sourced from these repositories.
	      \end{description}
	\item
	      \begin{description}
		      \item \textbf{critique - generalizability:} The proposal did not take into account the potential impact of different object types on their suitability for Boolean difference visualizations
		      \item \textbf{solution:} We have restructured our experiment to ensure that we evaluate a diverse array of objects sourced from real-world public repositories. Through conducting horizontal comparisons between various objects and monitoring results on an individual object basis, we aim to identify emerging patterns in the impact of object types on suitability.
	      \end{description}
\end{enumerate}

\section{Appendix B: Schedule}

Over the last couple of weeks, we have closely reviewed the paper motivating this work, reached out to the author, and followed up on the tools pointed out. A recurring theme among all tools is \texttt{three.js}, a 3D tool viewer.

% Split the problem up into 2 parts, first question, does this even really help. 2nd question, summarize the changes. We broke our research question up into two parts.

\subsection{Curation of Files}

We have identified some repositories that have a large community actively using OpenSCAD to modify 3D solids. We have identified an example of an open-sourced \texttt{onshape.com} project with full modification history.

\textbf{Over the next two weeks} we will curate these into a collection of folders with \emph{before} and \emph{after} files.

\subsection{Creation of Difference Visualizations}

We have implemented a simple \texttt{three.js} interface capable of loading basic \texttt{.stl} files.

\textbf{Over the next two weeks} we will explore the many plugins for \texttt{three.js} and implement the most appropriate. If \texttt{three.js} is inappropriate for this task, we will create them in another \texttt{.stl} modifying software like Blender or \texttt{onshape.com}.

% \section*{Appendix C: Egg man}

% \texttt{https://github.com/skalnik}


\end{document}
