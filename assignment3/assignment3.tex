%\documentclass[journal,12pt,onecolumn,]{IEEEtran}
\documentclass[]{IEEEtran}
% Chris, you can put this back before we submit.
% Seems like her students even publish single col.
% I can't stand it though, so for now, no options.
% Maybe we should use one of the various ACMs like the project?
% They all somehow manage to be even uglier than IEEE though. 
% At least the IEEE titles almost match the text... 

\usepackage{hyperref}

%%%%%Prof Zhou's Rant on Citations, Punctuation, and Quotes %%%%%%
%%%
%%% Yes - ...of seventy-two''~\cite{42}.
%%% Yes - ...of seventy-two.''\footnote{42}
%%%
%%% NO! - ...of seventy-two.''~\cite{42}.
%%%
%%% https://advice.writing.utoronto.ca/wp-content/uploads/sites/2/quotations.pdf
%%%
%%%%%%%%%%%%%%%%%%%%%%%%%%%%%%%%%%%%%%%%%%%%%%%%%%%%%%%%%%%%%%%%%%

\begin{document}

\title{Critique on \\
{\normalsize ECE1785: Assignment 3 - Intro Quantitative Analysis}}

\author{Amos Hebb, Yilong Wang, Zhenyue Yu\\ \small University of Toronto}

\maketitle

\section{Introduction}

In this assignment, we download 2017 Open Source Survey results.
5,500 respondents from over 3,800 repos, 500 non-random from other platforms.
Attitudes, experiences, and backgrounds of those who use, build, and maintain open-source software.

\subsection{RQ1: TODO}

Professional-software 'frequently' vs 'rarely/never' could be a proxy for hands on keyboards vs jira jockeys. Why else are they on github if they don't write code?
Okay got one: TRANSPARENCY-PRIVACY-BELIEFS. Hypothesis: doers are happy with anon, the MBAs want authorship.


EXTERNAL-EFFICACY - ``values contributions from people like me'', this feels fun, maybe see if the women hate it? (I'd shoot this down as a revealed preference problem but whatever)


INFO-AVAILABILITY,TRANSPARENCY-PRIVACY-PRACTICES-GENERAL - this one tickles my brain too. Who wants to be anon? why? I don't think this survey tells us enough to know.


Help sections? My (sexist?) hunch is that men are more likely to directly contact somebody for help while women may not be as comfortable doing so?
Too bad we only have like 0.3 female responses so even if we found something it'd not mean anything.

%TODO: Motivation

\subsection{RQ2: TODO}

%TODO: Motivation

\section{Data Cleaning}

Data are provided with a \textsc{README} and a \texttt{notes.txt} file.
\textsc{README} mostly details methodolgy, while \texttt{notes.txt} summarizes known data quality issues.
After filtering data according to these documents, we then TODO.

\subsection{\textsc{README}}

The \textsc{README} mostly contains Readme exploins that harassment related questions have been randomized.
% Okay so lets try to link them up anyway >:) Yeah no way, there's nothing here for us... Honestly, this whole negative section feels like a writeoff.

\subsection{\texttt{notes.txt}}

A document outlining six data quality issues ironically enumerated 1,2,2,3,4,5.
Based on the known issues with these data, and the substantial sample remaining, we have applied the following filters guided by the notes document.
The \texttt{NEGATIVE-WITNESS} family of columns were dropped, as they were not related to our research question.
Dates submitted are not considered.
We only consider rows where \texttt{TODO!!!.ANY.RESPONSE} is 0, \texttt{STATUS} is \texttt{Complete}, \texttt{POPULATION} is \texttt{github}.
We removed 82 \texttt{TRANSLATED} responses.
% I'm quite curious about these translated ones, but we have 3217 english, and 82 from 'traditional chinese, japanese, spanish, russian'. Curious choices, but almost certainly nothing representative to be extracted here.

\subsection{TODO: At least one nifty filter we applied ourselves}

\subsection{Unaddressed Issues}

Some data quality issues could not be addressed due to collection methodology.
\texttt{EMPLOYMENT.STATUS} is mutually exclusive, so part-time student and part-time employed is not possible.
\texttt{IMMIGRATION} does not include ``I intend to stay'' options for people living in the country where they were born.
The entire section on helping forces mutually exclusive choices and asks for most recent case where help took place instead of typical.
All questions were optional, so ``prefer not to say'' is presumably encoded by a non-response. Certain questions also have an explicit ``prefer not to say'' response, many of these also have low response rates.
Ages are bucketed unevenly within and different between the two age related questions.

\section{Statistical Analysis}

\section{Result}

We find a highly correlated correlate.

\section{Conclusion}

\bibliographystyle{IEEEtran}
\bibliography{assignment3}

\end{document}
