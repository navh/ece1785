\documentclass[conference]{IEEEtran}
\IEEEoverridecommandlockouts
% The preceding line is only needed to identify funding in the first footnote. If that is unneeded, please comment it out.
\usepackage{cite}
\usepackage{amsmath,amssymb,amsfonts}
\usepackage{algorithmic}
\usepackage{graphicx}
\usepackage{textcomp}
\usepackage{xcolor}
\begin{document}

\title{Comparing Emperical Methods\\
{\small ECE1785: Assignment 1 - Comparison of Methods}}

\author{\IEEEauthorblockN{1\textsuperscript{st} Given Name Surname}
    \IEEEauthorblockA{\textit{dept. name of organization (of Aff.)} \\
        \textit{name of organization (of Aff.)}\\
        City, Country \\
        email address or ORCID}
    \and
    \IEEEauthorblockN{2\textsuperscript{nd} Wang Yilong}
    \IEEEauthorblockA{\textit{dept. name of organization (of Aff.)} \\
        \textit{name of organization (of Aff.)}\\
        City, Country \\
        email address or ORCID}
    \and
    \IEEEauthorblockN{3\textsuperscript{rd} Given Name Surname}
    \IEEEauthorblockA{\textit{dept. name of organization (of Aff.)} \\
        \textit{name of organization (of Aff.)}\\
        City, Country \\
        email address or ORCID}
}

\maketitle

\begin{abstract}

\end{abstract}

\section{Introduction}

\section{How to Break an API}

The first paper we will consult is ``How to Break an API''~\cite{bogart2016break}, a 2016 multiple case study of three ecosystems philosophies toward change.

\subsection{Method}

A multiple case study, interviewing 28 developers in each of three ecosystems. Developers were recruited with recent, relevant experiences in similar roles contributing to packages with multiple upstream and downstream dependencies.
The three ecosystems studied are Eclipse, R/CRAN, and Node.js/npm.
Semi-structured interviews focused on personal practices and experiences negotiating upstream and downstream dependencies.
Interviews were recorded, transcribed, then coded.
Findings were validated with Dagenais and Robillard's~\cite{dagenais2010creating} methodology to check fit and applicability as defined by Corbin and Strauss~\cite{corbin2014basics}.

\subsection{Research Questions}

\begin{itemize}
    \item How do developers make decisions about whether and when to perform breaking changes?
    \item How do they mitigate or delay costs for other developers?
    \item How do developers react and manage change in their dependencies?
    \item How do policies, tooling, and community values influence decision-making?
\end{itemize}

\subsection{Evidence used to Reach Conclusion}


\section{Need for tweet}

The second paper we studied is "Need for tweet: How open source developers talk about their github work on twitter"~\cite{fang2020need} .

\subsection{Method}

\subsection{Research Question}

\subsection{Evidence used to reach conclusion}

\section{Differences in Methods}

\section{Differences between qualitative and quantitative}

\section{When each method is appropriate}

\subsection{Qualitative}

\subsection{Quantitative}

\section{Weaknesses of each method}

\subsection{Qualitative}

\subsection{Quantitative}

\section{Conclusion}

\bibliographystyle{IEEEtran}
\bibliography{assignment1}

\end{document}
